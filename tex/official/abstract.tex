% $Log: abstract.tex,v $
% Revision 1.1  93/05/14  14:56:25  starflt
% Initial revision
% 
% Revision 1.1  90/05/04  10:41:01  lwvanels
% Initial revision
% 
%
%% The text of your abstract and nothing else (other than comments) goes here.
%% It will be single-spaced and the rest of the text that is supposed to go on
%% the abstract page will be generated by the abstractpage environment.  This


Since the definition of the Busy Beaver function by Rad\'{o} in 1962, an interesting open question has been what the smallest value of $n$ for which $BB(n)$ is independent of ZFC. Is this $n$ approximately 10, or closer to 1,000,000, or is it unfathomably large? In this thesis, I show that it is at most \statenum by presenting an explicit description of a \statenumstate Turing machine $Z$ with 1 tape and a 2-symbol alphabet whose behavior cannot be proved in ZFC, assuming ZFC is consistent. The machine is based on work of Harvey Friedman on independent statements involving order-invariant graphs. \cite{friedman}~
In doing so, I give the first known upper bound on the highest provable Busy Beaver number in ZFC. I also present an explicit description of a \gbstatenumstate Turing machine $G$ that halts if and only if there's a counterexample to Goldbach's conjecture, and an explicit description of a \rmstatenumstate Turing machine $R$ that halts if and only if the Riemann hypothesis is false. In the process of creating $G$, $R$, and $Z$, I define a higher-level language, TMD, which is much more convenient than direct state manipulation, and explain in great detail the process of compiling this language down to a Turing machine description. TMD is a well-documented language that is optimized for parsimony over efficiency. This makes TMD a uniquely useful tool for creating small Turing machines that encode mathematical statements. 
