To see that this is true, consider the statement $S$: ``This statement is not provable in ZFC.'' $S$ cannot be proved with ZFC, since if it was proven then it would be false! But if it cannot be proved with ZFC, then it must be true. Hence, there is a true statement in ZFC that cannot be proved in ZFC. \\

To verify this, consider a scenario in which ZFC could prove its own consistency. Now suppose that the statement $S$ above was false. That would imply that $S$ \emph{is} provable in ZFC--which would mean ZFC can prove a falsehood, which is impossible! It follows that the statement $S$ above must be true. But if this conclusion follows from the assumption of ZFC's consistency, then it must be true that if ZFC can prove its own consistency, it can also prove $S$. However, it was established from the proof of the G\"{o}del's first incompleteness theorem that ZFC cannot prove $S$ and remain consistent! Thus, if ZFC is sound and can prove its own consistency, it must be able to prove $S$, so it must be inconsistent. But if ZFC is inconsistent and can prove its own consistency, it must not be sound! Therefore, assuming ZFC is sound, ZFC cannot prove its own consistency. J. Rosser was able to retain the proof while weakening the assumption of soundness to an assumption of consistency in 1936. \\ % TODO cite?

In the presented Turing machine, the special halting states \texttt{ACCEPT} and \texttt{REJECT} are used. However, there is an additional halting state that was not previously mentioned: the \texttt{ERROR} state. The \texttt{ERROR} state is not, strictly speaking, necessary to the Turing machines created by my compiler; it is, however, useful for debugging. The idea of the \texttt{ERROR} state is that the Turing machine should \emph{never enter} it. Situations which should be impossible point to the \texttt{ERROR} state, although in principle they could point to any state while preserving the same functionality. In addition to helping catch bugs in the compiler (because no compiled TMD program that does not throw an error when interpreted should even enter an \texttt{ERROR} state), the \texttt{ERROR} state helps keep lower-level transformation steps parsimonious, as will be described in Sections~\ref{sec:mttost} and~\ref{sec:mstots}. Section~\ref{sec:turdtotm} will give examples of the use of the \texttt{ERROR} states. \\

